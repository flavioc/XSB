\chapter{Getting Started with XSB} \label{quick_start}
%=====================================================

This section describes the steps needed to install XSB under UNIX and
under Windows.

\section{Installing XSB under UNIX}
%==================================
\label{installation_options}

If you are installing on a UNIX platform, the version of XSB that you
received may not include all the object code files so that an
installation will be necessary.  The easiest way to install XSB is to
use the following procedure.

\begin{enumerate}
\item	Decide in which directory in your file system you want to install
  XSB and copy or move XSB there.
\item Make sure that after you have obtained XSB by anonymous ftp (using
  the {\tt bin}ary option) or from the web, you have uncompressed it by
  following the instructions found in the file {\tt README}.
  
\item Note that after you uncompress and untar the XSB tar file, a
  subdirectory {\tt XSB} will be tacked on to the current directory. All
  XSB files will be located in that subdirectory.
  
  In the rest of this manual, let us use {\tt \$XSB\_DIR} to refer to this
  subdirectory.  Note the original directory structure of XSB must be
  maintained, namely, the directory {\tt \$XSB\_DIR} should contain all the
  subdirectories and files that came with the distribution. In particular,
  the following directories are required for XSB to work: \verb'emu',
  \verb'syslib', \verb'cmplib', \verb'lib', \verb'packages', \verb'build',
  and \verb'etc'.

\index{configuration}

\item Change directory to {\tt \$XSB\_DIR/build} and then run these commands:
  %%
  \begin{quote}
    \tt
    configure\\
    \tt
    makexsb
  \end{quote}
  %%
  %%$
  This is it!
  
  In addition, it is now possible to install XSB in a shared directory
  ({\it e.g.}, {\tt /usr/local}) for everyone to use.  In this situation,
  you should use the following sequence of commands:
  %%
  \begin{quote}
    \tt
    configure --prefix=\$SHARED\_XSB\\
    \tt
    makexsb\\
    \tt
    makexsb install
  \end{quote}
  %%$
  where {\tt \$SHARED\_XSB}  denotes the shared directory where XSB is
  installed.  In all cases, XSB can be run using the script
  %%$
  \begin{quote}
    \tt
    \$XSB\_DIR/bin/xsb
  \end{quote}
  However, if XSB is installed in a central location, the script for
  general use is:
  \begin{quote}
    \verb'<central-installation-directory>/<xsb-version>/bin/xsb'
  \end{quote}
\end{enumerate}
  %%

{\bf Important:} The XSB executable determines the location of the
libraries it needs based on the full path name by which it was invoked.
The ``smart script'' \verb|bin/xsb| also uses its full path name to
determine the location of the various scripts that it needs in order to
figure out the configuration of your machine.  Therefore, there are certain
limitations on how XSB can be invoked.

Here are some legal ways to invoke XSB:
%%
\begin{enumerate}
\item  invoking the smart script \verb|bin/xsb| or the XSB executable using
  their absolute or relative path name.
\item using an alias for \verb|bin/xsb| or the executable.
\item creating a new shell script that invokes either \verb|bin/xsb| or the
  XSB executable using their {\em full\/} path names. 
\end{enumerate}
%%

Here are some ways that are guaranteed to not work in some or all cases:
%%
\begin{enumerate}
\item  creating a hard link to either \verb|bin/xsb| or the executable and
  using {\it it\/} to invoke XSB. (Symbolic links should be ok.)
\item changing the relative position of either \verb|bin/xsb| or the
  XSB executable with respect to the rest of the XSB directory tree.
\end{enumerate}
%%

%%
\begin{description}
\item[Type of Machine.]  The configureation script automatically detects
  your machine and OS type, and builds XSB accordingly. Moreover, you can
  build XSB for different architectures while using the same tree and the
  same installation directory provided, of course, that these machines are
  sharing this directory, say using NFS or Samba. All you will have to do
  is to login to a different machine with a different architecture or OS
  type, and repeat the above sequence of comands.
  
  The configuration files for different architectures reside in different
  directories, and there is no danger of an architecture conflict.
  Moreover, you can keep using the same {\tt ./bin/xsb} script regardless
  of the architecture. It will detect your configuration and will use the
  right files for the right architecture! 
  
\item[Choice of the C Compiler and Other options] \label{cc}
  The {\tt configure} script will attempt to use {\tt gcc}, if it is available.
  Otherwise, it will revert to {\tt cc} or {\tt acc}.  Some versions of
  {\tt gcc} are broken, in which case you would have to give {\tt
    configure} an additional directive {\tt --with-cc}.  You can also {\tt
    --disable-optimization} (to change the default), {\tt --enable-debug},
  and there are many other options.  Type {\tt configure --help} to see
  them all. Also see the file \verb'$XSB_DIR/INSTALL' for more details.
    %%$
\end{description}

Other options are of interest to advanced users who wish to experiment
with XSB, or to use XSB for large-scale projects.  In general, however
users need not concern themselves with these options.

\begin{description}
\index{scheduling strategy}
\item[Type of Scheduling Strategy.]  The ordering of operations within
a tabled evaluation can drastically affect its performance.  XSB
provides two scheduling strategies: Batched Evaluation and Local
Evaluation.  Batched Evaluation is the default scheduling strategy for
XSB and evaluates queries to reduce the time to the first answer of a
query.  Local Evaluation can be chosen via the {\tt
--enable-local-scheduling} configure option.  Detailed explanations
can be found in \cite{JFLP-Scheduling}.
%
\index{memory management}
\index{garbage collection}
\item[Type of Memory Management.]  Routines for managing execution
stacks for tabled evaluations can be quite complex, due to
interdependencies of tabled subgoals.  Indeed, memory management
algorithms can be based on common elements are shared among
computation states or are copied.  The default configuration of XSB
shares these elements while the option {\tt --enable-chat} copies
these elements.  While sharing and copying have minor performance
differences, the main reason to try the {\tt --enable-chat}
configuration is to use a heap garbage collector that has been written
for it.  See \cite{SaSw98,CAT@PLILP-98,CATmem@ISMM-98,CHAT@PADL-99}
for in-depth discussion of the engine memory management.
\end{description}

\subsection{Possible Installation Problems}

\paragraph*{Lack of Space for Optimized Compilation of C Code}
When making the optimized version of the emulator, the temporary space
available to the C compiler for intermediate files is sometimes not
sufficient. For example on one of our SPARCstations that had very
little {\tt /tmp} space the {\tt "-O4"} option could not be used for
the compilation of files {\tt emuloop.c}, and {\tt tries.c}, without
changing the default {\tt tmp} directory and increasing the swap
space.  Depending on your C compiler, the amount and nature of {\tt
/tmp} and swap space of your machine you may or may not encounter
problems.  If you are using the SUN C compiler, and have disk space in
one of your directories, say {\tt dir}, add the following option to
the entries of any files that cannot be compiled:

\demo{       -temp=dir}

\noindent
If you are using the GNU C compiler, consult its manual pages
to find out how you can change the default {\tt tmp} directory or how you
can use pipes to avoid the use of temporary space during compiling.
Usually changing the default directory can be done by declaring/modifying
the {\tt TMPDIR} environment variable as follows:

\demo{       setenv TMPDIR dir}

\paragraph*{Missing XSB Object Files}
When an object (*.O) file is missing from the {\tt lib} directories you can
normally run the {\tt make} command  in that directory to restore it
(instructions for doing so are given in Chapter
\ref{quick_start}).  However, to restore an object file in the
directories {\tt syslib} and {\tt cmplib}, one needs to have a
separate Prolog compiler accessible (such as a separate copy of
XSB), because the XSB compiler uses most of the files in these
two directories and hence will not function when some of them are
missing.  For this reason, distributed versions normally include all
the object files in {\tt syslib} and {\tt cmplib}.

\section{Installing XSB under Windows}
\subsection{Using Cygnus Software's \mbox{CygWin32}}
\label{quick:cygwin}

This is easy: just follow the Unix instructions. This is the preferred way to
run XSB under Windows, because this ensures that all features of XSB are
available.


\subsection{Using Microsoft Visual C++}
\label{quick:DOS}
%==========================================

\begin{enumerate}
\item 
   XSB will unpack into a subdirectory named {\tt xsb}.
   Assuming that you have {\tt XSB.ZIP} in the {\tt \$XSB\_DIR} directory,
   you can issue the command
\begin{verbatim}
   unzip386 xsb.zip
\end{verbatim}
   which will install XSB in the subdirectory {\tt xsb}.

%% DOS is no longer supported!
%%\item 
%%   If you have no other source of DPMI support (Windows is one source),
%%   unpack {\tt csdpmi1b.zip} somewhere (unpacking in a directory named
%%   DJGPP is suggested), and be sure that the exe files are somewhere
%%   in your path.  Likewise for {\tt wmemu2b.zip}, if you don't have a
%%   floating-point unit.
%%
%%\item 
%%   If you modified {\tt AUTOEXEC.BAT}, either reboot or set the variables
%%   by hand.  Start XSB as follows:
%%\[
%%  \tt
%%   \$XSB\_DIR \backslash config\backslash \mbox{\tt
%%   x86-pc-pcdos}\backslash bin \backslash xsb
%%\]
   
 \item If you decide to move XSB to some other place, make sure that the
   entire directory tree is moved --- XSB executable looks for the files it
   needs relatively to its current position in the file system.

\end{enumerate}


You can compile XSB under Microsoft Visual C++ compiler 
to create a console-supported top loop or a DLL by following
these steps:

\begin{enumerate}

\item
   cd emu

\item
  Type (in one line):\\
  {\tt NMAKE /f "MS\_VC\_Mfile.mak" CFG="option" \\
    \hspace*{2cm}[DLL="yes"] [ORACLE="yes"] [SITE\_LIBS="libraries"]}
  %%
  \begin{itemize}
  \item The items in square brackets are optional.
  \item The options for {\tt CFG} are: \emph{release} or \emph{debug}.  The
    latter is used when you want to compile XSB with debugging enabled.
  \item The other parameters to {\tt NMAKE} are optional. The {\tt DLL}
    parameter tells Visual C++ to compile XSB as a DLL. The {\tt ORACLE}
    parameter compiles XSB with support for Oracle DBMS. If {\tt ORACLE} is
    specified, you {\bf must} also specify the necessary Oracle libraries
    using the parameter {\tt SITE\_LIBS}.
  \end{itemize}
  %%
   
 \item The above command will compile XSB as requested and will put the XSB 
   executable in:
%%
\[
 \tt
 \$XSB\_DIR\backslash config\backslash \mbox{\tt x86-pc-windows}\backslash bin
 \backslash xsb.exe
\]
%%
   If you requested to compile XSB as a DLL, then the DLL will be placed in
%%
\[
 \tt
 \$XSB\_DIR\backslash config\backslash \mbox{\tt x86-pc-windows}\backslash
 saved.o \backslash xsb.dll
\]
%%
\end{enumerate}
%%
{\bf Note}: the XSB executable and the DLL can coexist in the same source
tree structure. However, if you first compiled  XSB as an executable and
then want to compile it as a DLL (or vice versa), then you must run 
%%
\begin{verbatim}
 NMAKE /f "MS_VC_Mfile.mak" clean  
\end{verbatim}
%%
in between.


\section{Invoking XSB}
%=====================

Under Unix, XSB can be invoked by the command:
\begin{quote}
       \tt \$XSB\_DIR/bin/xsb
\end{quote}
%%$
if you have installed XSB in your private directory.
If XSB is instaled in a shared directory ({\it e.g.}, \$SHARED\_XSB
for the entire site (UNIX only), then you should use
\begin{quote}
       \tt \$SHARED\_XSB/bin/xsb
\end{quote}
%%
In both cases, you will find yourself in the top level interpreter.  
As mentioned above, this script automatically detects the system
configuration you are running on and will use the right files and
executables. (Of course, XSB should have been built for that architecture
earlier.)

Under Windows, you should invoke XSB by typing:
\[
 \tt
 \$XSB\_DIR\backslash config\backslash \mbox{\tt x86-pc-windows}\backslash bin
 \backslash xsb.exe
\]
%%


You may want to make an alias such as {\tt \smallourprolog} to the above
commands, for convenience, or you might want to put the directory where the
XSB command is found in the {\tt \$PATH} environment variable. However, you
should {\bf not} make hard links to this script or to the XSB executable.
If you invoke XSB via such a hard link, XSB will likely be confused and will
not find its libraries.  That said, you {\bf can} create other scripts and
cal the above script from there.

Most of the ``standard'' Prolog predicates are supported by XSB, 
so those of you who consider yourselves champion entomologists, can try
to test them for bugs now.  Details are in Chapter~\ref{standard}.


\section{Compiling XSB programs}
%===============================

All source programs should be in files whose names have the 
suffix {\tt .P}.  One of the ways to compile a program from a file in 
the current directory and load it into memory, is to type the query:
\begin{verbatim}
     [my_file].
\end{verbatim}
where \verb'my_file' is the name of the file, or preferably, the name
of the module (obtained from the file name by deleting the suffix {\tt .P}).
To find more about the module system of XSB see Section~\ref{Modules}.

If you are eccentric (or you don't know how to use an editor) you can also 
compile and load predicates input directly from the terminal by using the
command:
\begin{verbatim}
     [user].
\end{verbatim}
A {\tt CTRL-d} or the atom \verb'end_of_file' followed by a period 
terminates the input stream.


\section{Sample XSB Programs}
%============================

If for some reason you don't feel like writing your own XSB programs, 
there are several sample XSB programs in the directory: 
{\tt \$XSB\_DIR/examples}.  All contain source code.

The entry predicates of all the programs in that directory are given
the names {\tt demo/0} (which prints out results) and {\tt go/0}
(which does not print results).\footnote{
  %%
  This convention does not apply to the subdirectories of the examples
  directory, which illustrate advanced features of XSB.
  %%
  }
%%
Hence, a sample session might look like
(the actual times shown below may vary and some extra information is given
using comments after the \% character):

{\footnotesize
 \begin{verbatim}
my_favourite_prompt> cd $XSB_DIR/examples
my_favourite_prompt> $XSB_DIR/bin/xsb
XSB Version 2.0 (Gouden Carolus) of June 27, 1999
[i586-pc-linux-gnu; mode: optimal; engine: slg-wam; scheduling: batched]
| ?- [queens].
[queens loaded]

yes
| ?- demo.

% ...... output from queens program .......

Time used: 0.4810 sec

yes
| ?- statistics.

memory (total)         1906488 bytes:       203452 in use,      1703036 free
  permanent space       202552 bytes
  glob/loc space        786432 bytes:          432 in use,       786000 free
    global                                     240 bytes
    local                                      192 bytes
  trail/cp space        786432 bytes:          468 in use,       785964 free
    trail                                      132 bytes
    choice point                               336 bytes
  SLG subgoal space          0 bytes:            0 in use,            0 free
  SLG unific. space      65536 bytes:            0 in use,        65536 free
  SLG completion         65536 bytes:            0 in use,        65536 free
  SLG trie space             0 bytes:            0 in use,            0 free
   (call+ret. trie           0 bytes,     trie hash tables            0 bytes)

      0 subgoals currently in tables
      0 subgoal check/insert attempts inserted     0 subgoals in the tables
      0 answer  check/insert attempts inserted     0 answers  in the tables

       Time: 0.610 sec. cputime,  18.048 sec. elapsetime

yes
| ?- halt.          % I had enough !!!

End XSB (cputime 1.19 secs, elapsetime 270.25 secs)
my_favourite_prompt>
 \end{verbatim}
}


\section{Exiting XSB}
%====================

If you want to exit XSB, issue the command \verb'halt.' or
simply type \verb'CTRL-d' at the XSB prompt. To exit XSB while it is
executing queries, strike \verb'CTRL-c' a number of times.


%%% Local Variables: 
%%% mode: latex
%%% TeX-master: "manual1"
%%% End: 
