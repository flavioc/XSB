\chapter{The New XSB-Database Interface} \label{db_interface}
%====================================================

\begin{center}
{\Large {\bf By Saikat Mukherjee and Michael Kifer }}
\end{center}

\section{Introduction}
%=====================

The XSB-DB interface is a package that allows XSB users to access
databases through various drivers. Using this interface, information
in different DBMSs can be accessed by SQL queries. The interface defines
Prolog predicates which makes it easy to connect to databases, 
query them, and disconnect from the databases. Central to the 
concept of a connection to a database is the notion of a \emph{handle}.
A connection handle describes a particular connection to a database.
Similar to a connection handle is the notion of a query handle which
describes a particular query statement. As a consequence of the handles,
it is possible to open multiple database connections (to the same or
different databases) and keep alive multiple queries (again from the
same or different connections). The interface also supports dynamic loading of
drivers. As a result, it is possible to query databases using different drivers
concurrently.

We use the {\tt student} database as our example to illustrate 
the usage of XSB-DB interface in this manual. The schema of the
student database contains three columns viz. the student name, 
the student id, and the name of the advisor of the student.

\section{Using the Interface}
%========================================

The XSB-DB package has to be first loaded before using any of the
predicates. This is done by the call:

\begin{verbatim}
| ?- [dbdrivers].
\end{verbatim}

Next, the driver to be used for connecting to the database has to 
be loaded. Currently, the interface has support for a native MySQL driver
(using the MySQL C API), and an ODBC driver. For example, to load 
the ODBC driver call:

\begin{verbatim}
| ?- load_driver('odbc').
\end{verbatim}

Similarly, to load the mysql driver call:

\begin{verbatim}
| ?- load_driver('mysql').
\end{verbatim}

\subsection{Connecting to and Disconnecting from Databases}
%=================================================

There are two predicates for connecting to databases. 

\begin{verbatim}
| ?- db_connect(Handle, Driver, DSN, User, Password).
\end{verbatim}

This predicate assumes that an entry for a data source name (DSN) exists
in the {\tt ODBC.INI} file. The {\tt Handle} is the handle name used for 
the connection. The {\tt Driver} is the driver being used for the 
connection.
The {\tt User} and {\tt Password} are the user name and password 
being used for the connection. 
To connect to the data source mydb using the user name xsb and password xsb
with the odbc driver, the call is as follows: 

\begin{verbatim}
| ?- db_connect(ha, odbc, mydb, xsb, xsb).
\end{verbatim}

where ha is the handle name for the connection.

\begin{verbatim}
| ?- db_connect(Handle, Driver, Server, Database, User, Password).
\end{verbatim}

This predicate is used for a direct connection to a database i.e.
without the use of any dsn. The {\tt Handle}, {\tt Driver}, {\tt User},
and {\tt Password} are the same as before. The {\tt Server} and 
{\tt Database} arguments specify the server and database to connect to.
For example, for a connection to a database called test in the server
wolfe with the user name xsb, password xsb, and using the mysql
driver, the call is:

\begin{verbatim}
| ?- db_connect(ha, mysql, wolfe, test, xsb, xsb).
\end{verbatim}

where ha is the handle name for the connection.

If the connection is successfully made, the predicate invocation will
succeed.  This step is necessary before anything can be done with the
data sources since it gives XSB the opportunity to initialize system
resources for the session.

To close a database connection use:

\begin{verbatim}
| ?- db_disconnect(Handle).
\end{verbatim}

where handle is the handle name for the connection. For example, 
to close the connection to above mysql database call:

\begin{verbatim}
| ?- db_disconnect(ha).
\end{verbatim}

and XSB will give all the resources it allocated for this session back
to the system.


\subsection{Querying the database}
%=====================================

The interface supports two types of querying. In direct querying, the
query statement is not prepared while in prepared querying the query
statement is prepared before being executed. The results from
both types of querying are retrieved tuple at a time.
Direct querying is done by the predicate:

\begin{verbatim}
| ?- db_sql(ConnectionHandle, QueryHandle, SQLQueryList, ReturnList).
\end{verbatim}

ConnectionHandle is the name of the handle used for the database connection.
QueryHandle is the name of the handle for this particular query. 
Currently, the query handle is being used only for prepared queries. 
However, in future versions, the query handle can be used in direct
queries to retrieve arbitrary tuples from a result set using cursors.
Also, it will be possible to combine arbitrary tuples from different 
queries to the same database using the query handle.
The SQLQueryList is a list of terms which is used to build the SQL query.
The terms in this list are ground atoms. ReturnList is a list of 
variables each of which correspond to a return value in the query.
It is upto the user to specify the correct number of return variables 
corresponding to the query.
For example, a query on the student database to select all the students
for a given advisor is accomplished by the call:

\begin{verbatim}
| ?- X = adv, db_sql(ha, qa, ['select T.name from student T where 
T.advisor = ', X], [P]), fail.
\end{verbatim}

Observe that the query list is composed of the sql string and a ground value
for the advisor. The return list is made of one variable corresponding to
the student name. The failure drive loop retrieves all the tuples.

Preparing a  query is done by the call to the predicate:

\begin{verbatim}
| ?- db_prepare(ConnectionHandle, QueryHandle, SQLQueryList).
\end{verbatim}

As before, ConnectionHandle and QueryHandle specify the handles for
the connection and the query. The SQLQueryList is a list of terms which
build up the query string. The placeholder `?' is used for values which 
have to be bound during the execution of the statement.
For example, to prepare a query for selecting the advisor name for a student
name using our student database:

\begin{verbatim}
| ?- db_prepare(ha, qa, ['select T.advisor from student T where T.name = ?']).
\end{verbatim}

A prepared statement is executed using the predicate:

\begin{verbatim}
| ?- db_prepare_execute(QueryHandle, BindList, ReturnList).
\end{verbatim}

The BindList contains the ground values corresponding to the `?' in
the prepared statement. The ReturnList is a list of variables for
each argument in a tuple of the result set.

For direct querying, the statement handle is closed automatically when
all the tuples in the result set have been retrieved. In order to explicitly
close a statement handle, and free all the resources associated with
the handle, a call is made to the predicate:

\begin{verbatim}
| ?- db_statement_close(QueryHandle).
\end{verbatim}

where QueryHandle is the query handle for the statement to be closed.

\subsection{Error Handling}
%============================

Each predicate in the XSB-DB interface throws an exception with the functor name
{\tt xsb\_error} which contains a single term with the error message.
It is upto the user to catch this exception
and proceed with error handling. This is done by the throw-catch error
handling mechanism in XSB.


%%% Local Variables: 
%%% mode: latex
%%% TeX-master: "manual2"
%%% End: 
